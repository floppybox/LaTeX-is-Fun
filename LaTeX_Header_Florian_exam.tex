\documentclass[11pt,a4paper,addpoints,color,answers]{exam}	%	Package for typesetting exam scripts 


%%%%%%%%%%%%%%%%%%%%%%%%%%%%%%%%
%%%%% Anpassungen für exam %%%%%
%%%%%%%%%%%%%%%%%%%%%%%%%%%%%%%%
\usepackage{anyfontsize}
%\usepackage{color, colortbl}							% leider zerlegt es da meine tabularx-Umgebung...
\usepackage{xcolor}										% Colour control for LATEX documents
%\shadedsolutions
\renewcommand{\solutiontitle}{\noindent}
\SolutionEmphasis{\color{blue}}
\colorgrids
\definecolor{GridColor}{gray}{0.8}
\setlength{\gridsize}{5.03mm}
\setlength{\gridlinewidth}{0.3mm}
\setlength\dottedlinefillheight{5mm}
\colorfillwithdottedlines
\definecolor{FillWithDottedLinesColor}{gray}{0.5}

%%%%%%%%%%%%%%%%%%%%%%%%%%%%%%%%
%%%%% Anpassungen für exam %%%%%
%%%%%%%%%%%%%%%%%%%%%%%%%%%%%%%%


%%%%% \usepackage %%%%%
	%\usepackage{ae}                   			% Virtual fonts for T1 encoded CMR-fonts
	\usepackage{amsmath}						% AMS mathematical facilities for LATEX
	\usepackage{amsfonts}						% TEX fonts from the American Mathematical Society
	\usepackage{amssymb}						% z. B. für \leqq
	%\usepackage{bbm}                  			% "Blackboard-style" cm fonts
	\usepackage{chemformula}                    % Command for typesetting chemical formulas and reactions
	\usepackage{courier}						% Adobe Type 1 "free" copies of Courier
	\usepackage{hyperref}						% Extensive support for hypertext in LATEX
		\hypersetup{
			colorlinks	= true,
			linkcolor 	= black,
			urlcolor	= blue		
		}
	\usepackage[hyphenbreaks]{breakurl}			% Line-breakable \url-like links in hyperref when compiling via dvips/ps2pdf
	\usepackage[europeanresistors,americanvoltages,emptydiodes,siunitx]{circuitikz}		
												% Draw electrical networks with TikZ
	\usepackage[dvips]{epsfig}					% Include Encapsulated PostScript in LATEX documents
	\usepackage[left]{eurosym}    				% METAFONT and macros for Euro sign
	\usepackage{listings}						% Typeset source code listings using LATEX
		\lstset{literate=						% darstellen von Umlauten mit dem Paket listings
			{Ö}{{\"O}}1
			{Ä}{{\"A}}1
			{Ü}{{\"U}}1
			{ß}{{\ss}}1
			{ü}{{\"u}}1
			{ä}{{\"a}}1
			{ö}{{\"o}}1
		}
		\lstdefinestyle{DOS}
		{	
			backgroundcolor=\color{black},		% choose the background color
			basicstyle=\linespread{1.5}\scriptsize\color{white}\ttfamily,
			numbers=left,					% where to put the line-numbers
			numberstyle=\color{red},
			stepnumber=1,
			showspaces=false,
			tabsize=1,
			breaklines=true,
			breakatwhitespace=true,
			columns=fullflexible				% to avoid spaces between letters
		}
		\lstdefinestyle{DOS2}
		{	
			backgroundcolor=\color{white},		% choose the background color
			basicstyle=\linespread{1}\tiny\color{black}\ttfamily,
			numbers=left,					% where to put the line-numbers
			numberstyle=\color{red},
			stepnumber=1,
			showspaces=false,
			tabsize=1,
			breaklines=true,
			breakatwhitespace=true%,
%			columns=fullflexible				% to avoid spaces between letters
		}

	\usepackage{dsfont}							% Type­set math­e­mat­i­cal dou­ble stroke sym­bols
	\usepackage{natbib}
	\usepackage{graphicx}
	
	\usepackage{afterpage}
	
	\usepackage{epigraph}
	\setlength\epigraphwidth{.8\textwidth}
	\setlength\epigraphrule{0pt}
	

	%\usepackage{exscale}						% Implements scaling of the 'cmex' fonts
	%\usepackage{fp}							% Fixed point arithmetic
	\usepackage{fp}
	\usepackage{pgfkeys,pgfmath,pgfcore}
	
	\pgfkeys{
		/textnumber/.style={
			/pgf/number format/.cd,% <- changes the prefix for the following options
			fixed,
			use comma,
			fixed zerofill,
			precision=1,
			1000 sep={.},
		},
	}


	\usepackage{float} 							% Improved interface for floating objects \begin{figure}[H]	Places the float at precisely the location in the LaTeX code.
	\usepackage[T1]{fontenc}					% Standard package for selecting font encodings
	\usepackage{german}							% Support for German typography
	\usepackage{isotope}						% A package for typesetting isotopes
	\usepackage[utf8]{inputenc}					% Accept different input encodings
	%\usepackage{latexsym}             			% z. B. f\"{u}r den Befehl $\$\leadsto$$
%	\usepackage{makecell}						% Tabular column heads and multilined cells
	\usepackage{mathtools}						% Mathematical tools to use with amsmath (\intertext{and}; \shortintertext{and})
	\usepackage[version=4]{mhchem}				% Typeset chemical formulae/equations
	%\usepackage{picins}						% Insert pictures into paragraphs
	\usepackage{pst-all}
	%\usepackage{pstricks}
	%\usepackage{pstricks-add}
	\usepackage{pst-circ}						% A PSTricks package for drawing electric circuits. Wird benötigt für \psgrid
	%\usepackage[dvips]{rotating}				% Rotation tools, including rotated full-page floats

	\usepackage{titlesec}
	\newcommand{\sectionbreak}{\clearpage}

	\usepackage{siunitx}						% A comprehensive (SI) units package
		\sisetup{output-decimal-marker = {,}}	% Komma als Dezimaltrennzeichen für siunitx
		\sisetup{exponent-product = \cdot}		% Punkt statt x zwischen Zahl und Zehnerpotenz
		\sisetup{per-mode = fraction}			% Bruchstrich in Einheit
		\sisetup{inter-unit-separator ={}\cdot{}}
		\DeclareSIUnit\Ah{\text{Ah}}
		\DeclareSIUnit\mAh{\text{mAh}}
%		\DeclareSIUnit{\litre}{\ell}
		\DeclareSIUnit{\liter}{\ell}

	\usepackage{tabularx}
	\newcolumntype{Y}{>{\centering\arraybackslash}X}
	
	\usepackage{tikz}							% Create PostScript and PDF graphics in TEX
	\usepackage[europeanresistors,americanvoltages,emptydiodes,siunitx]{circuitikz}
		\usetikzlibrary{circuits.ee.IEC}		% Schaltbilder mit TikZ
		\usetikzlibrary{shapes,arrows,positioning,calc,backgrounds}
		\usetikzlibrary{
			circuits.logic.US,
			circuits.logic.IEC,
			circuits.logic.CDH,
			circuits.ee.IEC,
		}
		\usetikzlibrary{decorations.markings}
		\usetikzlibrary{arrows}
		\usetikzlibrary{decorations.pathmorphing}
		\usetikzlibrary{decorations.pathreplacing}			% für geschweifte Klammern
		\usetikzlibrary{decorations.text}		
		\tikzset{circuit declare symbol = ammeter}
		\tikzset{set ammeter graphic ={draw,generic circle IEC, minimum size=5.5mm,info=center:A}}
		
		\tikzset{circuit declare symbol = voltmeter}
		\tikzset{set voltmeter graphic ={draw,generic circle IEC, minimum size=5.5mm,info=center:V}}

%%%%% geschweifte Klammern von A nach B

		\newcommand{\tikzmark}[2][-3pt]{\tikz[remember picture, overlay, baseline=-0.5ex]\node[#1](#2){};}
		
		\tikzset{brace/.style={decorate, decoration={brace}},
			brace mirrored/.style={decorate, decoration={brace,mirror}},
		}
		
		\newcounter{brace}
		\setcounter{brace}{0}
		\newcommand{\drawbrace}[3][brace]{%
			\refstepcounter{brace}
			\tikz[remember picture, overlay]\draw[#1] (#2.center)--(#3.center)node[pos=0.5, name=brace-\thebrace]{};
		}
		
		\newcounter{arrow}
		\setcounter{arrow}{0}
		\newcommand{\drawcurvedarrow}[3][]{%
			\refstepcounter{arrow}
			\tikz[remember picture, overlay]\draw (#2.center)edge[#1]node[coordinate,pos=0.5, name=arrow-\thearrow]{}(#3.center);
		}
		
		% #1 options, #2 position, #3 text 
		\newcommand{\annote}[3][]{%
			\tikz[remember picture, overlay]\node[#1] at (#2) {#3};
		}

%%%%% geschweifte Klammern von A nach B
		

%%%%% geschweifte Klammern in Tabellen - Aternative		

		\usepackage{array, multirow, bigdelim, makecell, booktabs}
			%& \hspace{-10em}\rdelim\}{3.5}{*}[\quad $\Rightarrow W_\text{th}  = \SI{4,182}{\kilo\joule}$] 

%%%%% geschweifte Klammern in Tabellen - Aternative		



		\usepackage{url}							% Verbatim with URL-sensitive line breaks
		\usepackage{ulem}
	
%		\usepackage{version}						% Conditionally include text
%			\includeversion{gitter}
%			\excludeversion{gitter} 				% Auskommentieren, um die Gitter anzuzeigen
%			\includeversion{loesung}
	%\usepackage{yhmath}						% Extended maths fonts for LATEX
%	\usepackage{verbatim}						% Reimplementation of and extensions to LATEX verbatim
%%%%% \usepackage %%%%%

%%%%% \definecolor %%%%%
	\definecolor{dgreen}{rgb}{0,0.502,0}
	\definecolor{lgreen}{rgb}{0.753,1,0.753}
	\definecolor{lred}{rgb}{1,0.753,0.753}
	\definecolor{lblue}{rgb}{0.9216,0.9216,1}
	\definecolor{llblue}{rgb}{0.9608,0.9608,1}
% 	\newrgbcolor{lmagenta}{1 0.502 1}
%%%%% \definecolor %%%%%

%%%%% \newcommand %%%%%
	\newcommand{\aufgabe}{\vspace*{1ex}\stepcounter{aufg}\setcounter{enumi}{\value{aufg}}\setcounter{enumii}{0}\addtocounter{enumii}{-1}}
	%\newcommand*{\bbm}[1]{$\mathbbm{#1}$}
	%\newcommand*{\bbmi}[2]{$\mathbbm{#1}_#2$}
	\newcommand{\bx}[1]{\fbox{\rule[-0.5ex]{0ex}{3ex} \hspace*{#1ex}}}
	\newcommand{\bluebox}[1]{\textcolor{blue}{\fbox{\rule[-1ex]{0ex}{3.5ex} \quad #1 \quad}}}
	\newcommand*\circled[1]{\tikz[baseline=(char.base)]{
			\node[shape=circle,draw,inner sep=1.2pt] (char) {#1};}}
	\newcommand{\D}{\displaystyle}
	\newcommand{\grid}[1]{\\[1ex]		
		\smallskip
		\psset{unit=0.5cm}\hfill
		\begin{pspicture}(32,#1)
		\psgrid[griddots=10,gridwidth=0.2pt,gridlabels=0pt,%
		subgriddiv=0,subgriddots=10,gridcolor=black](0,0)(0,0)(32,#1)
		\end{pspicture}}
	%\newcommand*{\ie}{i.\,e. }
	%\newcommand*{\kmph}{$\frac{\mbox{km}}{\mbox{h}}$}
	%\newcommand*{\kmps}{$\frac{\mbox{km}}{\mbox{s}}$}
	\newcommand{\kosy}[5]{
		\begin{center}
			\tikzstyle{background grid}=[draw, black!20,step=5mm,line width=1pt]
			\begin{tikzpicture}[scale=1.0,samples=400%
%			,show background grid
			]
				
				%% Größe des KOSYs %%%%%%%%%%%%%%%%%%%%%%%%%%%%%%%%%%%%%%%%%%
				
				\pgfmathsetmacro{\xmin}{#1}
				\pgfmathsetmacro{\xmax}{#2}
				\pgfmathsetmacro{\ymin}{#3}
				\pgfmathsetmacro{\ymax}{#4}
				
				%%%%%%%%%%%%%%%%%%%%%%%%%%%%%%%%%%%%%%%%%%%%%%%%%%%%%%%%%%%%%
				
				\pgfmathtruncatemacro{\xmintick}{\xmin}
				\pgfmathtruncatemacro{\xmaxtick}{\xmax}
				\pgfmathtruncatemacro{\ymintick}{\ymin}
				\pgfmathtruncatemacro{\ymaxtick}{\ymax}
				
				\pgfmathtruncatemacro{\xmingrid}{\xmin-1}
				\pgfmathtruncatemacro{\xmaxgrid}{\xmax+1}
				\pgfmathtruncatemacro{\ymingrid}{\ymin-1}
				\pgfmathtruncatemacro{\ymaxgrid}{\ymax+1}
				
				\pgfmathsetmacro{\xminkosy}{\xmin-0.5}
				\pgfmathsetmacro{\xmaxkosy}{\xmax+0.5}
				\pgfmathsetmacro{\yminkosy}{\ymin-0.5}
				\pgfmathsetmacro{\ymaxkosy}{\ymax+0.5}
				
				% Definition des Gitters und das Gitter selbst
				
				\tikzstyle{dotted}= [dash pattern=on 0.15mm off 0.85mm]
				\draw[help lines,line width=0.15mm,step=0.5,style=dotted,black,line cap=round] (\xmingrid,\ymingrid) grid (\xmaxgrid,\ymaxgrid);
				
				% Ursprung
				
				\node[below left] (0,0) {O};
				
				% Beschriftung der Koordinatenachsen mit Ticks (muss vor den Achsen selbst kommen, sonst sind die Achsen z. T. abgedeckt)
				
				\foreach \x in {\xmintick,...,-1,1,2,...,\xmaxtick}
				{
					\draw node[anchor=north,fill=white, fill opacity=0.7] at (\x,0) {\x};
					\draw (\x,0pt) -- (\x,-2pt) node[anchor=north] at (\x,0) {\x};
				}		
				
				\foreach \y in {\ymintick,...,-1,1,2,...,\ymaxtick}
				{
					\draw node[anchor=east,fill=white, fill opacity=0.7] at (0,\y) {\y};
					\draw (0pt,\y) -- (-2pt,\y) node[anchor=east] at (0,\y) {\y};
				}
				
				% Koordinatenachsen und Beschriftung mit $x$ und $y$
				
				\draw[-latex,line width=0.2mm] (\xminkosy,0) --(\xmaxkosy,0) node[below]{$x$};   
				\draw[-latex,line width=0.2mm] (0,\yminkosy) --(0,\ymaxkosy) node[left]{$y$};
				
				#5
				
			\end{tikzpicture}
		\end{center}
	}

\newcommand{\kosycustomi}[9]{
	\begin{center}
		\begin{tikzpicture}[scale=1,samples=400]
		
		%% Größe des KOSYs %%%%%%%%%%%%%%%%%%%%%%%%%%%%%%%%%%%%%%%%%%
		
		\pgfmathsetmacro{\xmin}{#1}
		\pgfmathsetmacro{\xmax}{#2}
		\pgfmathsetmacro{\ymin}{#3}
		\pgfmathsetmacro{\ymax}{#4}
		
		%%%%%%%%%%%%%%%%%%%%%%%%%%%%%%%%%%%%%%%%%%%%%%%%%%%%%%%%%%%%%
		
		\pgfmathtruncatemacro{\xmintick}{1+\xmin}
		\pgfmathtruncatemacro{\xmaxtick}{\xmax}
		\pgfmathtruncatemacro{\ymintick}{1+\ymin}
		\pgfmathtruncatemacro{\ymaxtick}{\ymax}
		
		\pgfmathtruncatemacro{\xmingrid}{\xmin-1}
		\pgfmathtruncatemacro{\xmaxgrid}{\xmax+2}
		\pgfmathtruncatemacro{\ymingrid}{\ymin-1}
		\pgfmathtruncatemacro{\ymaxgrid}{\ymax+1}
		
		\pgfmathsetmacro{\xminkosy}{\xmin-0.5}
		\pgfmathsetmacro{\xmaxkosy}{\xmax+0.5}
		\pgfmathsetmacro{\yminkosy}{\ymin-0.5}
		\pgfmathsetmacro{\ymaxkosy}{\ymax+0.5}
		
		% Definition des Gitters und das Gitter selbst
		
		\tikzstyle{dotted}= [dash pattern=on 0.15mm off 0.85mm]
		\draw[help lines,line width=0.15mm,step=0.5,style=dotted,black!20,line cap=round] (\xmingrid,\ymingrid) grid (\xmaxgrid,\ymaxgrid);
		
		% Ursprung
		
		\node[below left=2pt] (0,0) {\footnotesize 0};
		
		% Beschriftung der Koordinatenachsen mit Ticks (muss vor den Achsen selbst kommen, sonst sind die Achsen z. T. abgedeckt)
		
		\foreach \x in {\xmintick,...,1,2,3,...,\xmaxtick}
		{
			\pgfmathsetmacro\resultx{#7*\x}
			%			\draw node[anchor=north,fill=white, fill opacity=0.9] at (\x,0) {\footnotesize \resultx};
			\draw (\x,0pt) -- (\x,-2pt) node[anchor=north] at (\x,-0.1) {\footnotesize \pgfmathprintnumber[/textnumber]{\resultx}};
		}		
		
		\foreach \y in {\ymintick,...,1,2,3,...,\ymaxtick}
		{
			\pgfmathsetmacro\resulty{#8*(\y)}
			%			\draw node[anchor=east,fill=white, fill opacity=0.9] at (0,\y) {\footnotesize \resulty};
			\draw (0pt,\y) -- (-2pt,\y) node[anchor=east] at (-0.1,\y) {\footnotesize \pgfmathprintnumber[/textnumber]{\resulty}};
		}
		
		% Koordinatenachsen und Beschriftung mit $x$ und $y$
		
		\draw[-latex,line width=0.2mm] (\xminkosy,0) --(\xmaxkosy,0) node[right]{#5};   
		\draw[-latex,line width=0.2mm] (0,\yminkosy) --(0,\ymaxkosy) node[above]{#6};
		
		#9
		
		\end{tikzpicture}
	\end{center}
}	
	
	%\newcommand*{\linie}{
	%	
	%	\vspace*{1ex}
	%	
	%	\hrulefill}
	%\newcommand*{\mmpd}{$\frac{\mbox{mm}}{\mbox{d}}$}
	%\newcommand*{\mph}{$\frac{\mbox{m}}{\mbox{h}}$}
	%\newcommand*{\mps}{$\frac{\mbox{m}}{\mbox{s}}$}
	%\newcommand*{\ph}{ & \quad \fbox{\rule[-0ex]{0ex}{2ex} \hspace*{1.5ex}} \quad }
	\newcommand{\pu}{\marginpar{(\hspace{3.25ex}/\hspace{3ex})}}
	\newcommand{\pun}[1]{   \marginpar{\mbox{(\hspace{3.25ex}|\;\,\sffamily#1\;\,)}}
		\addtocounter{punkte}{#1}}
	\newcommand{\redcheck}{\ensuremath{\textcolor{red}{\checkmark}}}
%	\newcommand{\redcheckm}{\textcolor{red}{\checkmark}}
	\newcommand{\rueckseite}{
		\vfill
		
		\begin{flushright}
			Auf der Rückseite geht es weiter!
		\end{flushright}
		
		\newpage
		\thispagestyle{empty} \vspace*{-2em}}
	\newcommand{\naechsteseite}{
	\vfill
	
	\begin{flushright}
		Auf der nächsten Seite geht es weiter!
	\end{flushright}
	
	\newpage
	\thispagestyle{empty} \vspace*{-2em}}
	%\newcommand{\shortgrid}[3]{
	%	
	%	\smallskip
	%	\psset{unit=0.5cm}\hfill
	%	\begin{pspicture}(#2,#1)
	%	\psgrid[griddots=10,gridwidth=0.2pt,gridlabels=0pt,%
	%	subgriddiv=0,subgriddots=10,gridcolor=black](0,0)(0,0)(#3,#1)
	%	\end{pspicture}}
	\newcommand*{\zb}{z.\,B. }
	
	% https://tex.stackexchange.com/questions/78929/underlining-an-equation-in-an-align-block
	
	\makeatletter
	\tikzset{%
		remember picture with id/.style={%
			remember picture,
			overlay,
			save picture id=#1,
		},
		save picture id/.code={%
			\edef\pgf@temp{#1}%
			\immediate\write\pgfutil@auxout{%
				\noexpand\savepointas{\pgf@temp}{\pgfpictureid}}%
		},
		if picture id/.code args={#1#2#3}{%
			\@ifundefined{save@pt@#1}{%
				\pgfkeysalso{#3}%
			}{
				\pgfkeysalso{#2}%
			}
		}
	}
	
	\def\savepointas#1#2{%
		\expandafter\gdef\csname save@pt@#1\endcsname{#2}%
	}
	
	\def\tmk@labeldef#1,#2\@nil{%
		\def\tmk@label{#1}%
		\def\tmk@def{#2}%
	}
	
	\tikzdeclarecoordinatesystem{pic}{%
		\pgfutil@in@,{#1}%
		\ifpgfutil@in@%
		\tmk@labeldef#1\@nil
		\else
		\tmk@labeldef#1,(0pt,0pt)\@nil
		\fi
		\@ifundefined{save@pt@\tmk@label}{%
			\tikz@scan@one@point\pgfutil@firstofone\tmk@def
		}{%
			\pgfsys@getposition{\csname save@pt@\tmk@label\endcsname}\save@orig@pic%
			\pgfsys@getposition{\pgfpictureid}\save@this@pic%
			\pgf@process{\pgfpointorigin\save@this@pic}%
			\pgf@xa=\pgf@x
			\pgf@ya=\pgf@y
			\pgf@process{\pgfpointorigin\save@orig@pic}%
			\advance\pgf@x by -\pgf@xa
			\advance\pgf@y by -\pgf@ya
		}%
	}
	\makeatother
	
	\newcommand{\tikzmarkindouble}[2][]{%
		\tikz[remember picture,overlay,baseline=1ex]
		\draw[line width=0.5pt,#1,double]
		(pic cs:#2) -- (0,0)
		;}
	
	\newcommand{\tikzmarkin}[2][]{%
		\tikz[remember picture,overlay,baseline=1ex]
		\draw[line width=0.5pt,#1]
		(pic cs:#2) -- (0,0)
		;}
	
	\newcommand\tikzmarkend[2][]{%
		\tikz[remember picture with id=#2,baseline=1ex] #1;}

	\newcommand{\UpArrow}{\mathord{~\begin{tikzpicture}[baseline=0ex, line width=0.4, scale=0.13, ->, >=latex]
			\draw (0,0) -- (0,2);
			\end{tikzpicture}}}
	
	\newcommand{\DownArrow}{\mathord{~\begin{tikzpicture}[baseline=0ex, line width=0.4, scale=0.13, <-, >=latex]
			\draw (0,0) -- (0,2);
			\end{tikzpicture}}}
		
	\tikzset{dependent/.style={annotation arrow/.style = {>=}}}
	
	%LowDep %%%%%
	\tikzset{LowDep/.style args={#1}{
			append after command={%
				\bgroup
				[current point is local=true]
				[every LowDep/.try]
				[annotation arrow,-]
				(-2.5\tikzcircuitssizeunit,-1.5\tikzcircuitssizeunit) edge[line to]
				(-1.5\tikzcircuitssizeunit,-1.5\tikzcircuitssizeunit) node[xshift=3.0\tikzcircuitssizeunit]{#1}
				\egroup%
		}},
		%
		LowDep'/.style args={#1}{
			append after command={%
				\bgroup
				[current point is local=true, yscale=-1]
				[every LowDep/.try]
				[annotation arrow,-]
				(-2.5\tikzcircuitssizeunit,-1.5\tikzcircuitssizeunit) edge[line to]
				(-1.5\tikzcircuitssizeunit,-1.5\tikzcircuitssizeunit) node[xshift=3.0\tikzcircuitssizeunit]{#1}
				\egroup%
		}}
	}
	
	%UpDep %%%%%
	\tikzset{UpDep/.style args={#1}{
			append after command={%
				\bgroup
				[current point is local=true]
				[every UpDep/.try]
				[annotation arrow,-]
				%
				(2.5\tikzcircuitssizeunit,1.5\tikzcircuitssizeunit)  edge[line to]
				(1.5\tikzcircuitssizeunit,1.5\tikzcircuitssizeunit) node[xshift=-3.0\tikzcircuitssizeunit]{#1}
				\egroup%
		}},
		%
		UpDep'/.style args={#1}{
			append after command={%
				\bgroup
				[current point is local=true, yscale=-1]
				[every UpDep/.try]
				[annotation arrow,-]
				%
				(2.5\tikzcircuitssizeunit,1.5\tikzcircuitssizeunit)  edge[line to]
				(1.5\tikzcircuitssizeunit,1.5\tikzcircuitssizeunit) node[xshift=-3.0\tikzcircuitssizeunit]{#1}
				\egroup%
		}}
	}
	
\newcommand*\mycirc[1]{%
	\begin{tikzpicture}[baseline=(C.base)]
	\node[draw,circle,inner sep=1pt](C) {#1};
	\end{tikzpicture}}

	
%%%%% \newcommand %%%%%

%%%%% \newcounter %%%%%
	\newcounter{nullpunkte}
	\newcounter{punkte}[nullpunkte]
	\newcounter{aufg}\setcounter{aufg}{0}
%%%%% \newcounter %%%%%

%%%%% newenvironment %%%%%
\newenvironment{flo}{\begin{list}{\addtocounter{enumi}{-1}\arabic{enumi}\hspace{1.8ex}}{\usecounter{enumi} \setlength{\labelsep}{4mm}\setlength{\leftmargin}{26pt}}{}{}}{\end{list}\vspace*{-0.75em}}
\newenvironment{flo.}{\begin{list}{\arabic{enumi}.\arabic{enumii}}{\usecounter{enumii} \setlength{\labelsep}{4mm}\setlength{\leftmargin}{26pt}}{}{}}{\end{list}\vspace*{-0.75em}}
%%%%% newenvironment %%%%%

%%%%% newfont %%%%%
	\newfont{\calligraphic}{callig15 scaled 1200}
%%%%% newfont %%%%%

%%%%% \renewcommand %%%%%
	%\renewcommand*{\euro}{{}\,\officialeuro{}}
	%\renewcommand*{\SI}[2]{\ensuremath{\SI{#1}{#2}}} % klappt nicht u d ist evt. auch nicht notwendig?
	\renewcommand*{\dh}{d.\,h. }
	\renewcommand{\questionshook}{%
		\setlength{\labelsep}{4mm}%{20.5pt}
		\setlength{\leftmargin}{26pt}
	}
	
	\renewcommand{\partshook}{%
		\setlength{\labelsep}{4mm}
		\setlength{\leftmargin}{0pt}
	}
	
	\renewcommand{\choiceshook}{%
		\setlength{\labelsep}{20.5pt}
		\setlength{\leftmargin}{26pt}
	}
%%%%% \renewcommand %%%%%


%%%%% \setlength %%%%%
	\setlength{\parindent}{0em}       % kein Erstzeileneinzug
	\setlength{\textheight}{282mm} 
	\setlength{\textwidth}{167mm}
	\setlength{\oddsidemargin}{-13mm} 
	\setlength{\topmargin}{-28mm}
	\setlength{\marginparsep}{8mm}
	\setlength{\rightpointsmargin}{3mm}
%%%%% \setlength %%%%%

%%%%% exam %%%%%
\pointformat{\fontsize{10}{12}\selectfont (\quad\,\,\,|\,\,\,\textsf{\themarginpoints}\,\,\,)}
%\pointformat{\fontsize{10}{12}\selectfont (\quad\,\,\,|\,\,\,\textsf{\textcolor{white}{\themarginpoints}}\,\,\,)}
%%%%% exam %%%%%

