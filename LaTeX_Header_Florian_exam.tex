\documentclass[11pt,a4paper,addpoints,color,answers]{exam}	%	Package for typesetting exam scripts 


%%%%%%%%%%%%%%%%%%%%%%%%%%%%%%%%
%%%%% Anpassungen für exam %%%%%
%%%%%%%%%%%%%%%%%%%%%%%%%%%%%%%%
\usepackage{anyfontsize}
\usepackage{color}							% Colour control for LATEX documents
%\shadedsolutions
\renewcommand{\solutiontitle}{\noindent}
\SolutionEmphasis{\color{blue}}
\colorgrids
\definecolor{GridColor}{gray}{0.8}
\setlength{\gridsize}{5.03mm}
\setlength{\gridlinewidth}{0.3mm}
\setlength\dottedlinefillheight{5mm}
\colorfillwithdottedlines
\definecolor{FillWithDottedLinesColor}{gray}{0.5}

%%%%%%%%%%%%%%%%%%%%%%%%%%%%%%%%
%%%%% Anpassungen für exam %%%%%
%%%%%%%%%%%%%%%%%%%%%%%%%%%%%%%%


%%%%% \usepackage %%%%%
	%\usepackage{ae}                   			% Virtual fonts for T1 encoded CMR-fonts
	\usepackage{amsmath}						% AMS mathematical facilities for LATEX
	\usepackage{amsfonts}						% TEX fonts from the American Mathematical Society
	\usepackage{amssymb}						% z. B. für \leqq
	%\usepackage{bbm}                  			% "Blackboard-style" cm fonts
	\usepackage{courier}						% Adobe Type 1 "free" copies of Courier
	\usepackage{hyperref}						% Extensive support for hypertext in LATEX
		\hypersetup{
			colorlinks	= true,
			linkcolor 	= black,
			urlcolor	= blue		
		}
	\usepackage[hyphenbreaks]{breakurl}			% Line-breakable \url-like links in hyperref when compiling via dvips/ps2pdf
	\usepackage[europeanresistors,americanvoltages,emptydiodes,siunitx]{circuitikz}		
												% Draw electrical networks with TikZ
	\usepackage[dvips]{epsfig}					% Include Encapsulated PostScript in LATEX documents
	\usepackage[left]{eurosym}    				% METAFONT and macros for Euro sign
	\usepackage{listings}						% Typeset source code listings using LATEX
		\lstset{literate=						% darstellen von Umlauten mit dem Paket listings
			{Ö}{{\"O}}1
			{Ä}{{\"A}}1
			{Ü}{{\"U}}1
			{ß}{{\ss}}1
			{ü}{{\"u}}1
			{ä}{{\"a}}1
			{ö}{{\"o}}1
		}
		\lstdefinestyle{DOS}
		{	
			backgroundcolor=\color{black},		% choose the background color
			basicstyle=\linespread{1.5}\scriptsize\color{white}\ttfamily,
			numbers=left,					% where to put the line-numbers
			numberstyle=\color{red},
			stepnumber=1,
			showspaces=false,
			tabsize=1,
			breaklines=true,
			breakatwhitespace=true,
			columns=fullflexible				% to avoid spaces between letters
		}
		\lstdefinestyle{DOS2}
		{	
			backgroundcolor=\color{white},		% choose the background color
			basicstyle=\linespread{1}\tiny\color{black}\ttfamily,
			numbers=left,					% where to put the line-numbers
			numberstyle=\color{red},
			stepnumber=1,
			showspaces=false,
			tabsize=1,
			breaklines=true,
			breakatwhitespace=true%,
%			columns=fullflexible				% to avoid spaces between letters
		}

	\usepackage{dsfont}							% Type­set math­e­mat­i­cal dou­ble stroke sym­bols
	\usepackage{natbib}
	\usepackage{graphicx}
	
	\usepackage{afterpage}
	
	\usepackage{epigraph}
	\setlength\epigraphwidth{.8\textwidth}
	\setlength\epigraphrule{0pt}
	

	%\usepackage{exscale}						% Implements scaling of the 'cmex' fonts
	%\usepackage{fp}							% Fixed point arithmetic
	\usepackage{float} 							% Improved interface for floating objects \begin{figure}[H]	Places the float at precisely the location in the LaTeX code.
	\usepackage[T1]{fontenc}					% Standard package for selecting font encodings
	\usepackage{german}							% Support for German typography
	\usepackage{isotope}						% A package for typesetting isotopes
	\usepackage[utf8]{inputenc}					% Accept different input encodings
	%\usepackage{latexsym}             			% z. B. f\"{u}r den Befehl $\$\leadsto$$
	\usepackage{makecell}						% Tabular column heads and multilined cells
	\usepackage{mathtools}						% Mathematical tools to use with amsmath (\intertext{and}; \shortintertext{and})
	\usepackage[version=4]{mhchem}				% Typeset chemical formulae/equations
	%\usepackage{picins}						% Insert pictures into paragraphs
	\usepackage{pst-all}
	%\usepackage{pstricks}
	%\usepackage{pstricks-add}
	\usepackage{pst-circ}						% A PSTricks package for drawing electric circuits. Wird benötigt für \psgrid
	%\usepackage[dvips]{rotating}				% Rotation tools, including rotated full-page floats

	\usepackage{titlesec}
	\newcommand{\sectionbreak}{\clearpage}

	\usepackage{siunitx}						% A comprehensive (SI) units package
		\sisetup{output-decimal-marker = {,}}	% Komma als Dezimaltrennzeichen für siunitx
		\sisetup{exponent-product = \cdot}
	\usepackage{tikz}							% Create PostScript and PDF graphics in TEX
	\usepackage[europeanresistors,americanvoltages,emptydiodes,siunitx]{circuitikz}
		\usetikzlibrary{circuits.ee.IEC}		% Schaltbilder mit TikZ
		\usetikzlibrary{shapes,arrows,positioning,calc}
		\usetikzlibrary{
			circuits.logic.US,
			circuits.logic.IEC,
			circuits.logic.CDH,
			circuits.ee.IEC,
		}
		\usetikzlibrary{decorations.markings}
		\usetikzlibrary{arrows}
		\usetikzlibrary{decorations.pathmorphing}
		\usetikzlibrary{decorations.pathreplacing}			% für geschweifte Klammern
		\usepackage{url}							% Verbatim with URL-sensitive line breaks
		\usepackage{ulem}
		
%		\usepackage{version}						% Conditionally include text
%			\includeversion{gitter}
%			\excludeversion{gitter} 				% Auskommentieren, um die Gitter anzuzeigen
%			\includeversion{loesung}
	%\usepackage{yhmath}						% Extended maths fonts for LATEX
%	\usepackage{verbatim}						% Reimplementation of and extensions to LATEX verbatim
%%%%% \usepackage %%%%%

%%%%% \definecolor %%%%%
	\definecolor{dgreen}{rgb}{0,0.502,0}
	\definecolor{lgreen}{rgb}{0.753,1,0.753}
	\definecolor{lred}{rgb}{1,0.753,0.753}
	\definecolor{lblue}{rgb}{0.9216,0.9216,1}
	\definecolor{llblue}{rgb}{0.9608,0.9608,1}
% 	\newrgbcolor{lmagenta}{1 0.502 1}
%%%%% \definecolor %%%%%

%%%%% \newcommand %%%%%
	\newcommand{\aufgabe}{\vspace*{1ex}\stepcounter{aufg}\setcounter{enumi}{\value{aufg}}\setcounter{enumii}{0}\addtocounter{enumii}{-1}}
	%\newcommand*{\bbm}[1]{$\mathbbm{#1}$}
	%\newcommand*{\bbmi}[2]{$\mathbbm{#1}_#2$}
	\newcommand{\bx}[1]{\fbox{\rule[-0.5ex]{0ex}{3ex} \hspace*{#1ex}}}
	%\newcommand*\circled[1]{\tikz[baseline=(char.base)]{
	%		\node[shape=circle,draw,inner sep=1.2pt] (char) {#1};}}
	%\newcommand{\D}{\displaystyle}
	\newcommand{\grid}[1]{\\[1ex]		
		\smallskip
		\psset{unit=0.5cm}\hfill
		\begin{pspicture}(32,#1)
		\psgrid[griddots=10,gridwidth=0.2pt,gridlabels=0pt,%
		subgriddiv=0,subgriddots=10,gridcolor=black](0,0)(0,0)(32,#1)
		\end{pspicture}}
	%\newcommand*{\ie}{i.\,e. }
	%\newcommand*{\kmph}{$\frac{\mbox{km}}{\mbox{h}}$}
	%\newcommand*{\kmps}{$\frac{\mbox{km}}{\mbox{s}}$}
	\newcommand{\kosy}[5]{
		\begin{center}
			\begin{tikzpicture}[scale=1.0,samples=400]
				
				%% Größe des KOSYs %%%%%%%%%%%%%%%%%%%%%%%%%%%%%%%%%%%%%%%%%%
				
				\pgfmathsetmacro{\xmin}{#1}
				\pgfmathsetmacro{\xmax}{#2}
				\pgfmathsetmacro{\ymin}{#3}
				\pgfmathsetmacro{\ymax}{#4}
				
				%%%%%%%%%%%%%%%%%%%%%%%%%%%%%%%%%%%%%%%%%%%%%%%%%%%%%%%%%%%%%
				
				\pgfmathtruncatemacro{\xmintick}{\xmin}
				\pgfmathtruncatemacro{\xmaxtick}{\xmax}
				\pgfmathtruncatemacro{\ymintick}{\ymin}
				\pgfmathtruncatemacro{\ymaxtick}{\ymax}
				
				\pgfmathtruncatemacro{\xmingrid}{\xmin-1}
				\pgfmathtruncatemacro{\xmaxgrid}{\xmax+1}
				\pgfmathtruncatemacro{\ymingrid}{\ymin-1}
				\pgfmathtruncatemacro{\ymaxgrid}{\ymax+1}
				
				\pgfmathsetmacro{\xminkosy}{\xmin-0.5}
				\pgfmathsetmacro{\xmaxkosy}{\xmax+0.5}
				\pgfmathsetmacro{\yminkosy}{\ymin-0.5}
				\pgfmathsetmacro{\ymaxkosy}{\ymax+0.5}
				
				% Definition des Gitters und das Gitter selbst
				
				\tikzstyle{dotted}= [dash pattern=on 0.15mm off 0.85mm]
				\draw[help lines,line width=0.15mm,step=0.5,style=dotted,black,line cap=round] (\xmingrid,\ymingrid) grid (\xmaxgrid,\ymaxgrid);
				
				% Ursprung
				
				\node[below left] (0,0) {O};
				
				% Beschriftung der Koordinatenachsen mit Ticks (muss vor den Achsen selbst kommen, sonst sind die Achsen z. T. abgedeckt)
				
				\foreach \x in {\xmintick,...,-1,1,2,...,\xmaxtick}
				{
					\draw node[anchor=north,fill=white, fill opacity=0.7] at (\x,0) {\x};
					\draw (\x,0pt) -- (\x,-2pt) node[anchor=north] at (\x,0) {\x};
				}		
				
				\foreach \y in {\ymintick,...,-1,1,2,...,\ymaxtick}
				{
					\draw node[anchor=east,fill=white, fill opacity=0.7] at (0,\y) {\y};
					\draw (0pt,\y) -- (-2pt,\y) node[anchor=east] at (0,\y) {\y};
				}
				
				% Koordinatenachsen und Beschriftung mit $x$ und $y$
				
				\draw[-latex,line width=0.2mm] (\xminkosy,0) --(\xmaxkosy,0) node[below]{$x$};   
				\draw[-latex,line width=0.2mm] (0,\yminkosy) --(0,\ymaxkosy) node[left]{$y$};
				
				\draw[line width=1pt,domain=0:#2+0.5] plot (\x,{#5});
				\draw[color=magenta,line width=1pt,->] (0,3.5) -- (1.693,3.5);
				\draw[color=magenta,line width=1pt,->] (1.693,3.5) -- (1.693,0);
				\draw node[anchor=west,fill=white, fill opacity=0.7] at (1.8,1) {$\textcolor{red}{\checkmark}$zu 3.5};				
				\draw node[anchor=west,fill=white, fill opacity=0.7] at (1,5) {$\textcolor{red}{\checkmark}\textcolor{red}{\checkmark}$};
				
			\end{tikzpicture}
		\end{center}
	}	
	%\newcommand*{\linie}{
	%	
	%	\vspace*{1ex}
	%	
	%	\hrulefill}
	%\newcommand*{\mmpd}{$\frac{\mbox{mm}}{\mbox{d}}$}
	%\newcommand*{\mph}{$\frac{\mbox{m}}{\mbox{h}}$}
	%\newcommand*{\mps}{$\frac{\mbox{m}}{\mbox{s}}$}
	%\newcommand*{\ph}{ & \quad \fbox{\rule[-0ex]{0ex}{2ex} \hspace*{1.5ex}} \quad }
	\newcommand{\pu}{\marginpar{(\hspace{3.25ex}/\hspace{3ex})}}
	\newcommand{\pun}[1]{   \marginpar{\mbox{(\hspace{3.25ex}|\;\,\sffamily#1\;\,)}}
		\addtocounter{punkte}{#1}}
	\newcommand{\redcheck}{\ensuremath{\textcolor{red}{\checkmark}}}
%	\newcommand{\redcheckm}{\textcolor{red}{\checkmark}}
	\newcommand{\rueckseite}{
		\vfill
		
		\begin{flushright}
			Auf der Rückseite geht es weiter!
		\end{flushright}
		
		\newpage
		\thispagestyle{empty} \vspace*{-2em}}
	\newcommand{\naechsteseite}{
	\vfill
	
	\begin{flushright}
		Auf der nächsten Seite geht es weiter!
	\end{flushright}
	
	\newpage
	\thispagestyle{empty} \vspace*{-2em}}
	%\newcommand{\shortgrid}[3]{
	%	
	%	\smallskip
	%	\psset{unit=0.5cm}\hfill
	%	\begin{pspicture}(#2,#1)
	%	\psgrid[griddots=10,gridwidth=0.2pt,gridlabels=0pt,%
	%	subgriddiv=0,subgriddots=10,gridcolor=black](0,0)(0,0)(#3,#1)
	%	\end{pspicture}}
	\newcommand*{\zb}{z.\,B. }
	
	% https://tex.stackexchange.com/questions/78929/underlining-an-equation-in-an-align-block
	
	\makeatletter
	\tikzset{%
		remember picture with id/.style={%
			remember picture,
			overlay,
			save picture id=#1,
		},
		save picture id/.code={%
			\edef\pgf@temp{#1}%
			\immediate\write\pgfutil@auxout{%
				\noexpand\savepointas{\pgf@temp}{\pgfpictureid}}%
		},
		if picture id/.code args={#1#2#3}{%
			\@ifundefined{save@pt@#1}{%
				\pgfkeysalso{#3}%
			}{
				\pgfkeysalso{#2}%
			}
		}
	}
	
	\def\savepointas#1#2{%
		\expandafter\gdef\csname save@pt@#1\endcsname{#2}%
	}
	
	\def\tmk@labeldef#1,#2\@nil{%
		\def\tmk@label{#1}%
		\def\tmk@def{#2}%
	}
	
	\tikzdeclarecoordinatesystem{pic}{%
		\pgfutil@in@,{#1}%
		\ifpgfutil@in@%
		\tmk@labeldef#1\@nil
		\else
		\tmk@labeldef#1,(0pt,0pt)\@nil
		\fi
		\@ifundefined{save@pt@\tmk@label}{%
			\tikz@scan@one@point\pgfutil@firstofone\tmk@def
		}{%
			\pgfsys@getposition{\csname save@pt@\tmk@label\endcsname}\save@orig@pic%
			\pgfsys@getposition{\pgfpictureid}\save@this@pic%
			\pgf@process{\pgfpointorigin\save@this@pic}%
			\pgf@xa=\pgf@x
			\pgf@ya=\pgf@y
			\pgf@process{\pgfpointorigin\save@orig@pic}%
			\advance\pgf@x by -\pgf@xa
			\advance\pgf@y by -\pgf@ya
		}%
	}
	\makeatother
	
	\newcommand{\tikzmarkindouble}[2][]{%
		\tikz[remember picture,overlay,baseline=1ex]
		\draw[line width=0.5pt,#1,double]
		(pic cs:#2) -- (0,0)
		;}
	
	\newcommand{\tikzmarkin}[2][]{%
		\tikz[remember picture,overlay,baseline=1ex]
		\draw[line width=0.5pt,#1]
		(pic cs:#2) -- (0,0)
		;}
	
	\newcommand\tikzmarkend[2][]{%
		\tikz[remember picture with id=#2,baseline=1ex] #1;}
	
%%%%% \newcommand %%%%%

%%%%% \newcounter %%%%%
	\newcounter{nullpunkte}
	\newcounter{punkte}[nullpunkte]
	\newcounter{aufg}\setcounter{aufg}{0}
%%%%% \newcounter %%%%%

%%%%% newenvironment %%%%%
\newenvironment{flo}{\begin{list}{\addtocounter{enumi}{-1}\arabic{enumi}\hspace{1.8ex}}{\usecounter{enumi} \setlength{\labelsep}{4mm}\setlength{\leftmargin}{26pt}}{}{}}{\end{list}\vspace*{-0.75em}}
\newenvironment{flo.}{\begin{list}{\arabic{enumi}.\arabic{enumii}}{\usecounter{enumii} \setlength{\labelsep}{4mm}\setlength{\leftmargin}{26pt}}{}{}}{\end{list}\vspace*{-0.75em}}
%%%%% newenvironment %%%%%

%%%%% newfont %%%%%
	\newfont{\calligraphic}{callig15 scaled 1200}
%%%%% newfont %%%%%

%%%%% \renewcommand %%%%%
	%\renewcommand*{\euro}{{}\,\officialeuro{}}
	\renewcommand*{\dh}{d.\,h. }
	\renewcommand{\questionshook}{%
		\setlength{\labelsep}{4mm}%{20.5pt}
		\setlength{\leftmargin}{26pt}
	}
	
	\renewcommand{\partshook}{%
		\setlength{\labelsep}{4mm}
		\setlength{\leftmargin}{0pt}
	}
	
	\renewcommand{\choiceshook}{%
		\setlength{\labelsep}{20.5pt}
		\setlength{\leftmargin}{26pt}
	}
%%%%% \renewcommand %%%%%


%%%%% \setlength %%%%%
	\setlength{\parindent}{0em}       % kein Erstzeileneinzug
	\setlength{\textheight}{282mm} 
	\setlength{\textwidth}{167mm}
	\setlength{\oddsidemargin}{-13mm} 
	\setlength{\topmargin}{-28mm}
	\setlength{\marginparsep}{8mm}
	\setlength{\rightpointsmargin}{3mm}
%%%%% \setlength %%%%%

%%%%% exam %%%%%
\pointformat{\fontsize{10}{12}\selectfont (\quad\,\,\,|\,\,\,\textsf{\themarginpoints}\,\,\,)}
%%%%% exam %%%%%

